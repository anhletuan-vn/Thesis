\chapter{Conclusion}




Limitation of current rdf engine ~\note{Distribution of Semantic Reasoning on the Edge of Internet of Things}

\nop{
%RISC-style DBMS design
Chaudhuri and Weikum [2000] used the term RISC in their proposal for different
self-tuning RISC-style database system architecture. Their presented concepts are
inspired from RISC-based central processing unit (CPU) design, which advocates
the construction of a CPU using simpler and faster instructions instead of complex
and slow instructions [Patterson and Ditzel, 1980]. Use of complex instruction for
CPU design has its own benefits, which includes upward compatibility and better
marketing opportunities. We believe similar benefits as a factor that derived existing
DBMS components towards prevailing complexity. However, Patterson and
Ditzel [1980] discussed both approaches in the context of CPU design and presented
the benefits of RISC-based design, such as simple and fast instruction, a possibility
of careful pruning of an instruction set, and minimizing complexity to maximize performance.
We intend to achieve similar benefits in our RISC-style Cellular DBMS
architecture.

Chaudhuri and Weikum [2000] suggested the use of RISC-style data managers with
narrow functionality, specialized API, small footprint, and limited interaction. Their
aim is to reduce the number of tuning knobs for a DBMS to make it more predictable
in terms of performance and behavior making it easy to self-tune. They defended
their proposal with the notion of “gain/pain” ratio, which suggests to tolerate a
moderate degradation in “gain” with the introduction of overheads present in their
approach to reduce the “pain” related to tuning with more predictable performance.

}